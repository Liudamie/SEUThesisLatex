\chapter{研究现状}
本文围绕微型无人机集群协同飞行应用场景,重点研究其关键基础技术,主要包括无人机集群协同定位与自主导航两个方面。
针对上述研究方向,本节将分别对国内外研究现状进行系统分析与总结。

\section{无人机集群协同定位}
相对定位使无人机集群中的每个个体都能够获取自身相对于他无人机的位置信息,从而为集群内的协调运动与协作任务执行提供基础支撑。
该能力对于维持编队结构、避免无人机间碰撞以及实现集群行为的一致性与同步性具有重要意义,进而显著提升无人机集群在复杂环境中的整体运行效率与任务执行能力。

近年来,针对无人机集群的相对定位问题,研究人员提出并应用了多种定位方法。其中,基于视觉感知的相对定位方法因其信息量丰富、定位精度较高而受到广泛关注\cite{lucahigh,yin2025iloc}。
然而,该类方法通常依赖视距条件,易受光照变化、遮挡和视场范围限制的影响,且在大规模或高密度无人机集群中难以实现稳定、全局的相对定位,计算与通信开销也随集群规模显著增加。

相比之下,基于无线通信的定位方法不依赖视距条件,具有覆盖范围广、对环境适应性强等优势,更适合用于无人机集群的相对定位任务。
其中,超宽带(Ultra-Wideband,UWB)无线电技术因其纳秒级时间分辨率和抗多路径能力强,已成为当前定位领域的重要研究方向,并被广泛应用于室内定位系统\cite{chiasson2023asynchronous,
yang2022vuloc,kim2022uwb,zhao2022finding,domuta2021two,cano2022clock,ayman2022calibration}
以及多机器人系统\cite{cano2023ranging, nguyen2023relative,jia2023distributed,liu2023relative}。
UWB能够在无人机之间的报文交换的过程中提供精确的发送与接收时间戳,从而支持基于飞行时间(Time of Flight,ToF)的距离测量,实现无人机间距离的连续估计与动态更新。

综上所述,基于超宽带的测距与协同定位方法已被广泛应用于微型无人机集群场景,并逐渐成为该领域的研究热点。
近年来,围绕超宽带测距机制与协同定位算法的相关研究不断涌现,有效推动了无人机集群协同定位技术的发展。
下文将分别从基于超宽带的测距技术和协同定位算法两个方面,对相关研究工作进行系统综述。

\subsection{基于超宽带的测距技术}


常见的双边双向测距(Double-Sided Two-Way Ranging,DS-TWR)协议最初是为一对一测距场景设计的。
为支持多对多同时测距操作,一种基于 DS-TWR 的简单扩展方案\cite{bux2005chapter}提出通过引入令牌环机制对测距过程进行协调。
尽管该方案在一定程度上提升了多节点间的测距能力,但其整体效率仍受到明显限制。
具体而言,在节点之间进行测距报文交互时,由于无线通信固有的广播特性,测距报文往往会被多个邻近节点接收。
然而,除目标节点外,其余邻居节点接收到的报文未被进一步利用,导致潜在可用报文被丢弃,从而造成通信资源浪费并降低系统整体测距效率。

针对多节点测距需求,新修订的 IEEE 802.15.4z-2020 标准\cite{9179124} 提出了基于 DS-TWR 的多对多测距协议。
该协议引入“测距轮”作为基本时间单位,并在每个测距轮内更新集群中节点的测距信息。
在协议规范性和工程可实现性方面对 DS-TWR 进行了有益扩展。
然而,该协议仍依赖基于时隙分配的测距机制,在高密度网络场景下其可扩展性受到显著限制。
在高密度网络中,单个设备或无人机往往需要与大量邻居节点逐一进行测距,而每一对测距节点通常需要占用多个时隙才能完成一次完整的测距过程。
随着网络密度的增加,完成一轮测距所需的时间显著增长,从而严重影响系统的实时性和可扩展性。

在无人机集群应用中,Guo 等人\cite{guo2017ultra}和 Li 等人\cite{li2020autonomous}分别在实际场景中验证了 UWB 技术在三架和五架无人机编队飞行中的可行性,其系统均采用双向测距(Two-Way Ranging,TWR)方法进行测距与通信。
然而,随着集群规模的增大,测距与通信频率显著下降,系统难以维持高效的信息交互,从而限制了其向大规模无人机集群协同飞行的扩展能力。

针对上述问题,亟需一种能够在大规模、高密度无人机集群中同时实现高效通信与稳定测距的 UWB 测距协议。
Shan 等人\cite{shan2022ultra,shan2021ultra}提出的集群测距协议在理论上具备实现高效测距与通信的潜力,但该协议最初并非面向相对定位任务设计,在实际应用中暴露出一定的鲁棒性问题。
例如,当测距消息以固定周期发送时,节点间时钟偏差可能导致持续的消息冲突,从而长时间无法完成距离计算;
而当测距消息以完全随机方式发送时,消息交换不匹配将不可避免,由于缺乏有效时间戳,距离计算难以进行。
在定位任务中,连续且频繁的测距失败将直接降低集群中各个无人机的定位精度,且在高密度集群中,该问题会因相邻节点数量增多而进一步加剧。


除面向高密度集群测距场景的研究外,基于超宽带的相关工作还广泛分布于其他室内定位应用场景,主要集中在测距协议设计、非视距(Non-Line-of-Sight,NLOS)环境处理以及测距误差补偿等方面。


在测距协议设计方面,Chiasson 等人 \cite{chiasson2023asynchronous} 提出了一种新的基于到达时间差(Time Difference of Arrival,TDOA)的双曲定位方程。该方法利用能够相互观测数据包到达时间的锚节点,对传统 TDOA 方程进行重构,从而有效降低了时钟漂移误差的影响,并在无需严格时间同步的条件下实现了高精度定位。
Yang 等人 \cite{yang2022vuloc}提出了虚拟双向测距方法,该方法支持大量目标的无同步定位,显著降低了系统对时间同步和校准的依赖。上述研究均在无需严格时间同步的前提下实现了高效测距,但通常依赖固定锚点部署。

在 NLOS 场景处理方面,Zhao 等人 \cite{zhao2022finding} 提出了一种新型算法,在存在障碍物的情况下,通过优化 UWB 锚点的部署位置来降低 NLOS 对定位精度的影响。
Kim 等人 \cite{kim2022uwb} 则利用长短期记忆网络对接收到的 UWB 信号信道脉冲响应进行信道状态分类,从而识别 NLOS 状态并提升定位性能。

在测距误差补偿方面,Domuta 等人 \cite{domuta2021two} 提出了两种额外的时间戳捕获机制以及一种新的飞行时间计算公式,用于系统性补偿节点间的时钟偏差。
Cano 等人 \cite{cano2022clock} 在此基础上进一步提出了一种仅依赖板载测量值的补偿方法,能够消除视距 ToF 测量中大部分观测偏差,并可靠地实现厘米级测距精度。
此外,Shalaby 等人 \cite{liu2023relative} 对双向测距的校准问题进行了深入研究,提出了一种全新的测距协议,结合稳健且可扩展的天线延迟校准程序,实现了多个 UWB 标签天线延迟的高效与精确校准。

综上所述,目前尚不存在能够在资源受限的大规模集群中支持大量节点同时进行测距的理想测距协议。现有研究大多面向节点数量有限或依赖固定锚点的应用场景,其协议设计通常基于时隙分配或集中式调度机制,在高密度、动态变化的无人机集群环境中难以同时兼顾测距效率、通信开销与系统鲁棒性。

\subsection{基于距离的相对定位算法}
高效稳定的定位方法是实现无人机之间协同感知与协调控制的关键基础,因此长期以来受到研究人员的广泛关注。
传统无人机的定位通常依赖于外部定位系统。例如,文献\citen{vasarhelyi2018optimized}中的无人机集群系统利用全球导航卫星系统(Global Navigation Satellite System,GNSS)实现对 30 架无人机的定位,从而成功展示了户外集群飞行行为。
然而,微型无人机往往在室内或封闭环境中运行,此类环境中 GNSS 等外部定位信号难以获取甚至完全不可用。
针对这一问题,已有研究提出了多种室内定位系统,通常依赖于预先部署的信标或锚点为空中集群提供定位信息 \cite{jia2022composite,oumar2023indoor,brandstatter2022multi}。
尽管这类方法在受限环境下能够实现较高精度的定位,但其对外部基础设施的依赖限制了在危险区域或未知环境中的应用,实际部署成本和灵活性均受到制约。

作为替代方案,基于无人机间无线通信的相对定位方法因其系统结构轻便、对机载资源需求较低而备受关注。
该类方法通过无线信道交换无人机的状态信息(如速度、偏航角速度和高度),并结合无线测距获取的相对距离信息来实现相对定位 \cite{nguyen2023relative,cao2021relative}。
文献 \citen{nguyen2023relative} 提出了基于优化的协同定位框架,包括二次约束方法及其半正定规划松弛形式。
尽管上述方法在机载无人机上能够获得较高的定位精度,但其单次定位计算时间约为 15 ms,对于计算资源和功耗受限的微型无人机而言仍然构成较大挑战。
Cao 等人 \cite{cao2021relative} 提出了一种利用多个超宽带测距节点估计无人机之间相对位姿的方法,但该方法通常依赖更大尺寸的平台和更强的计算能力,限制了其在微型无人机系统中的应用。
与上述基于优化或多节点测距的方法不同,基于滤波的协同定位方法通过递推状态估计来实现实时定位,其中扩展卡尔曼滤波器(Extended Kalman Filter,EKF)基于一阶线性化,因其结构简单、计算复杂度低,成为协同定位中最常用的状态估计方法之一\cite{li2020autonomous}。

为进一步提升协同定位的精度与系统鲁棒性,研究者相继提出了多种改进型非线性滤波方法,包括无迹卡尔曼滤波\cite{allotta2016development}、基于 Huber 损失的鲁棒滤波器\cite{bo2019cooperative}、 基于Student’s t 分布建模的鲁棒高斯近似滤波方法\cite{huang2017new}、粒子滤波算法\cite{akai2023reliable}以及移动视界估计方法\cite{wang2023neural}。
然而,上述协同定位算法的性能普遍对系统噪声和测量噪声协方差矩阵的先验设定高度敏感。当噪声统计特性建模存在偏差时,滤波器的估计精度将显著下降,严重情况下甚至可能引发滤波发散问题。
在微型无人机协同定位的复杂动态环境中,由于传感器误差、通信干扰及环境不确定性的共同影响,噪声协方差矩阵往往难以被准确刻画,这在一定程度上制约了现有方法在实际场景中的应用效果。

自适应卡尔曼滤波器(Adaptive Kalman Filter,AKF)被认为是解决无人机集群协同定位中噪声协方差矩阵未知问题的一种有效方法\cite{kruse2025adaptive}。
针对噪声统计特性未知的情形,已有大量相关研究工作,现有方法总体上可分为两类。

第一类自适应卡尔曼滤波方法通常假设过程噪声协方差矩阵已知,
主要致力于对测量噪声协方差矩阵的在线估计,
例如基于变分贝叶斯框架的自适应卡尔曼滤波方法\cite{davari2019variational}以及基于协方差匹配原理的自适应卡尔曼滤波方法\cite{hajiyev2023covariance}。
然而,此类方法对过程噪声协方差矩阵的先验信息依赖较强,当过程噪声的统计特性未知或存在建模偏差时,其滤波性能和适用范围将受到明显制约。

另一类自适应卡尔曼滤波方法针对过程噪声与测量噪声协方差矩阵均未知的情形展开研究。
其中,多模型自适应卡尔曼滤波通过并行运行多组卡尔曼滤波器来应对系统模型的不确定性,但其计算复杂度较高,难以满足实时性要求\cite{youn2020novel};
迭代扩展卡尔曼滤波则利用创新序列的白噪声特性对噪声协方差矩阵进行估计,然而在过程噪声和测量噪声同时未知的情况下,该方法易导致滤波发散,从而影响估计稳定性\cite{lin2022efficient}。

综上所述,现有自适应卡尔曼滤波方法在计算开销、滤波稳定性以及整体性能等方面仍存在一定局限,使其在微型无人机协同定位等对实时性和鲁棒性要求较高的应用场景中面临诸多挑战。

\section{无人机自主导航}
强化学习(Reinforcement Learning,RL)作为一种典型的序列决策方法,在无人机导航、路径规划与动态避障等问题中展现出显著优势。
与传统基于规则或局部优化的方法不同,RL通过与环境的持续交互,能够在长期回报最大化的目标下学习最优或近似最优策略,从而有效应对具有时序依赖性和不确定性的复杂决策任务\cite{ding2020introduction,clifton2020q}。
近年来,深度强化学习(Deep Reinforcement Learning,DRL)通过将深度神经网络与强化学习框架相结合,显著提升了传统 RL 在高维状态空间和连续动作空间下的表达与决策能力。得益于深度网络强大的特征提取能力,
DRL 在复杂感知输入(如激光雷达、视觉信息)和高动态环境中表现出优异性能,并在多项基准任务中达到了甚至超越人类水平的控制效果\cite{li2023deep,mnih2015human,tang2025deep}。
这使得 DRL 成为解决复杂自主导航问题的有力工具。

然而,现有多数 DRL 方法侧重于最大化未来累积收益的期望值,往往忽略罕见但后果严重的灾难性风险。在无人机等安全关键型机器人应用中,仅追求高期望回报难以满足实际需求,如何在不确定环境下进行风险感知与安全决策成为亟需解决的关键挑战。

风险敏感性强化学习的一种直观思路是基于最坏情况收益进行决策,但该策略往往导致过度保守的行为\cite{noorani2025risk}。
为在性能与安全性之间取得平衡,近年来的研究开始对未来收益的分布进行建模,并通过调节风险度量生成具有不同风险偏好的策略\cite{bellemare2023distributional}。尽管部分工作仅通过高斯分布的均值和方差来近似刻画收益不确定性,分布式强化学习能够进一步恢复累积收益的完整分布信息\cite{luis2024value},从而更准确地刻画决策过程中的随机性。其重要优势在于能够自然支持不同风险倾向策略的生成,在风险感知控制与安全决策中展现出较大潜力\cite{urpi2021risk, ma2020distributional}。

基于上述研究进展,本小节首先综述基于强化学习的导航方法,随后介绍分布式强化学习在无人机导航与协同决策中的研究现状及典型应用。
\subsection{基于强化学习的自主导航}

近年来,基于强化学习的无人机自主导航方法取得了快速发展,其优势主要体现在对复杂环境的良好泛化能力以及在不确定条件下所具备的鲁棒决策性能。
随着无人机应用逐步从仿真走向真实场景,如何在学习框架中合理刻画环境风险与感知不确定性,已成为提升自主导航系统安全性与可靠性的关键挑战。

围绕上述目标,研究者提出了多种基于强化学习的无人机导航与避障方法。文献 \citen{kahn2017uncertainty}通过神经网络预测未来多个时间步内发生碰撞的概率,并将其作为避障决策的重要依据,同时结合 MC-dropout\cite{folgoc2021mc} 与 bootstrap 方法\cite{peer2021ensemble} 对模型预测不确定性进行量化评估。
文献 \citen{oyewola2024deep} 引入长短期记忆网络\cite{krichen2025long} 对无人机历史运动信息进行建模,使系统能够在动态环境中刻画不确定性的时序演化特征。
此外,Fan等人 \cite{fan2020learning} 采用无模型策略网络直接完成动作选择,以降低对环境建模的依赖。
文献 Cho等人\cite{cho2014properties} 则利用门控循环单元预测局部观测的不确定性,并依据预测方差自适应调节随机策略的探索强度,从而在导航安全性与探索效率之间实现折中。

除了导航性能之外,基于学习的无人机系统在安全性方面的问题同样引起了广泛关注。基于可达性分析的方法能够为系统安全性提供严格的理论保证,然而这类方法在面对视觉、激光雷达等高维且信息丰富的传感器输入时往往难以有效建模;
同时,随着系统状态维度的增加,其计算复杂度迅速上升,从而限制了其在复杂无人机系统中的可扩展性\cite{gu2024review,brunke2022safe,yu2022towards}。

针对上述问题,部分研究引入判别式学习模型来构建安全预测器\cite{ma2020discriminative},通过评估当前状态或预测轨迹是否存在触发不安全行为的风险,并在检测到潜在危险时切换至预先设计的安全控制策略。
然而,这类基于判别式安全预测的方法仍然存在一定的局限性。首先,当系统在实际运行过程中遭遇训练阶段未覆盖的环境情形,或输入数据分布发生变化时,判别模型往往难以对全新输入作出可靠判断,甚至可能将潜在的不安全状态误判为安全状态。相比之下,基于统计建模的不确定性估计方法在面对未知环境时通常会表现出较高的不确定性,从而为系统提供更为保守且可信的风险度量。
其次,此类方法通常依赖于一个预先设计的安全控制器,并假设该控制器能够将系统从任意不安全状态中恢复。然而,在复杂动态环境或计算资源受限的无人机平台上,这一假设往往难以成立,从而限制了方法的实际适用性。

基于上述分析,在本文关注的应用场景中,风险度量与不确定性估计亟需在有限计算资源条件下实现高效、可靠的推断。为此,本文提出的方法通过对不确定性进行直接建模并加以利用,以引导策略行为的调整,而不依赖于判别式安全预测器或额外引入安全控制模块。
随着不确定性水平的增加,策略能够自然退化为更为谨慎的探索行为,从而在提升系统安全性与鲁棒性的同时,兼顾方法的自动化程度与计算效率,使其更适合部署于计算资源受限的无人机平台。

\subsection{分布式强化学习}

分布强化学习(Distributional Reinforcement Learning)近年来发展迅速,其核心思想是对价值函数的完整分布进行建模,而非仅关注期望值\cite{ma2025dsac,lowet2025opponent}。
由于收益分布蕴含的信息远超一阶统计量,利用分布信息能够支持更精细的决策,从而提升策略性能。近期研究还表明,类似的分布式价值表征机制可能与人类大脑中的决策过程相一致\cite{dabney2020distributional}。

分布强化学习已被成功应用于多种安全关键型场景,例如遮挡路口的自动驾驶\cite{liu2024attention} 以及移动无人机的室内导航\cite{wang2025adaptive}。
这类方法通常能够在训练阶段学习具有不同风险偏好的策略,并在部署时通过调节风险指标进行选择。然而,对于给定的任务环境,其风险倾向通常仍然被设定为固定值。

在实际复杂环境中,理想的风险水平应随环境状态动态变化。例如,飞行器在良好天气条件下巡航时可采取相对激进的策略,而在恶劣天气下降落时则需要更加保守的行为。这表明,风险倾向不仅与任务目标相关,还应根据环境反馈进行自适应调整,而实现风险偏好的在线、自动调节是构建智能自主系统的重要一步。

在方法层面,分布强化学习的早期代表是分类 DQN\cite{bellemare2023distributional},其具体原理是用一组事先选好的、固定的收益取值点,来近似真实但连续的收益概率分布。随后,分位数回归方法被提出以提高分布表示的灵活性\cite{zhou2020non}。
其中,分位数回归 DQN\cite{zhou2020non}通过在固定分位点上近似分位数函数来学习收益分布,而隐式分位数网络\cite{dabney2018implicit}  则进一步通过从均匀分布中采样分位数分数,利用神经网络隐式建模分位数函数,从而提升近似精度与表达能力。
隐式分位数网络 通常以沃瑟斯坦距离作为训练损失,用于度量分布之间的差异。

在安全关键应用中,已有研究将分布强化学习与安全约束相结合。例如,文献\citen{kamran2021minimizing} 将隐式分位数网络引入自动驾驶场景,通过风险规避型策略与安全保证机制的结合,实现了复杂交叉路口的安全决策;
在此基础上,Choi等人\cite{choi2021risk} 提出了一种支持多种风险敏感水平的移动机器人导航方法,使机器人能够在办公环境中执行安全导航任务。

尽管上述方法允许在不重新训练策略的情况下调整风险偏好,但其风险倾向在部署阶段通常仍保持固定,难以充分适应动态变化的环境不确定性。

\section{研究现状小结}


综上所述,在无人机协同定位方面,现有研究已从依赖外部基础设施的绝对定位方法,逐步发展到基于无线通信与测距的相对定位框架。基于优化的方法虽能获得较高精度,但其计算复杂度和资源消耗限制了在微型无人机平台上的实时应用;基于滤波的协同定位方法因计算高效而更具实用性,但其性能高度依赖于噪声统计特性的先验建模。尽管自适应卡尔曼滤波在一定程度上缓解了噪声协方差未知的问题,现有方法在计算开销、滤波稳定性以及复杂动态环境下的鲁棒性方面仍存在不足,难以同时满足微型无人机对实时性与可靠性的双重要求。

在无人机自主导航领域,强化学习尤其是深度强化学习凭借其在高维感知输入和复杂动态环境中的强大决策能力,已成为解决自主导航与避障问题的重要工具。然而,多数现有方法仍以期望收益最大化为目标,对罕见但高风险事件的刻画能力有限,在安全关键型应用中存在潜在隐患。分布式强化学习通过对累积收益分布进行建模,为风险敏感决策提供了更丰富的信息表达,并已在自动驾驶和无人机导航等安全场景中展现出良好潜力。然而,当前方法通常假设风险偏好在部署阶段保持固定,难以根据环境不确定性和任务状态进行动态调整。

因此,如何在计算资源受限的无人机平台上,同时实现高效可靠的协同定位、不确定性感知以及风险自适应决策,仍然是无人机自主系统面临的关键挑战。这也为本文从不确定性建模与风险自适应角度出发,构建兼顾安全性、鲁棒性与实时性的无人机协同感知与自主导航方法提供了明确的研究动机。