\chapter{研究现状}
本文围绕微型无人机集群协同飞行应用场景,重点研究其关键基础技术,主要包括无人机集群协同定位与自主导航两个方面。
针对上述研究方向,本节将分别对国内外研究现状进行系统分析与总结。

\section{无人机集群协同定位}
协同定位使无人机集群中的每个个体都能够获取自身相对于他无人机的位置信息,从而为集群内的协调运动与协作任务执行提供基础支撑。
该能力对于维持编队结构、避免无人机间碰撞以及实现集群行为的一致性与同步性具有重要意义,进而显著提升无人机集群在复杂环境中的整体运行效率与任务执行能力。

近年来,针对无人机集群的相对定位问题,研究人员提出并应用了多种定位方法。其中,基于视觉感知的相对定位方法因其信息量丰富、定位精度较高而受到广泛关注\cite{lucahigh,yin2025iloc}。
然而,该类方法通常依赖视距条件,易受光照变化、遮挡和视场范围限制的影响,且在大规模或高密度无人机集群中难以实现稳定、全局的相对定位,计算与通信开销也随集群规模显著增加。

相比之下,基于无线通信的定位方法不依赖视距条件,具有覆盖范围广、对环境适应性强等优势,更适合用于无人机集群的相对定位任务。
其中,超宽带(Ultra-Wideband,UWB)无线电技术因其纳秒级时间分辨率和抗多路径能力强,已成为当前定位领域的重要研究方向,并被广泛应用于室内定位系统\cite{chiasson2023asynchronous,
yang2022vuloc,kim2022uwb,zhao2022finding,domuta2021two,cano2022clock,ayman2022calibration}
以及多机器人系统\cite{cano2023ranging, nguyen2023relative,jia2023distributed,liu2023relative}。
UWB能够在无人机之间的报文交换的过程中提供精确的发送与接收时间戳,从而支持基于飞行时间(Time of Flight,ToF)的距离测量,实现无人机间距离的连续估计与动态更新。

综上所述,基于超宽带的测距与协同定位方法已被广泛应用于微型无人机集群场景,并逐渐成为该领域的研究热点。
近年来,围绕超宽带测距机制与协同定位算法的相关研究不断涌现,有效推动了无人机集群定位技术的发展。
下文将分别从基于超宽带的测距技术和协同定位算法两个方面,对相关研究工作进行系统综述。

\subsection{基于超宽带的测距技术}


常见的双边双向测距(Double-Sided Two-Way Ranging,DS-TWR)协议最初是为一对一测距场景设计的。
为支持多对多同时测距操作,一种基于 DS-TWR 的简单扩展方案\cite{bux2005chapter}提出通过引入令牌环机制对测距过程进行协调。
尽管该方案在一定程度上提升了多节点间的测距能力,但其整体效率仍受到明显限制。
具体而言,在节点之间进行测距报文交互时,由于无线通信固有的广播特性,测距报文往往会被多个邻近节点接收。
然而,除目标节点外,其余邻居节点接收到的报文未被进一步利用,导致潜在可用报文被丢弃,从而造成通信资源浪费并降低系统整体测距效率。

针对多节点测距需求,新修订的 IEEE 802.15.4z-2020 标准\cite{9179124} 提出了基于 DS-TWR 的多对多测距协议。
该协议引入“测距轮”作为基本时间单位,并在每个测距轮内更新集群中节点的测距信息。
在协议规范性和工程可实现性方面对 DS-TWR 进行了有益扩展。
然而,该协议仍依赖基于时隙分配的测距机制,在高密度网络场景下其可扩展性受到显著限制。
在高密度网络中,单个设备或机器人往往需要与大量邻居节点逐一进行测距,而每一对测距节点通常需要占用多个时隙才能完成一次完整的测距过程。
随着网络密度的增加,完成一轮测距所需的时间显著增长,从而严重影响系统的实时性和可扩展性。

在无人机集群应用中,Guo 等人\cite{guo2017ultra}和 Li 等人\cite{li2020autonomous}分别在实际场景中验证了 UWB 技术在三架和五架无人机编队飞行中的可行性,其系统均采用双向测距(Two-Way Ranging,TWR)方法进行测距与通信。
然而,随着集群规模的增大,测距与通信频率显著下降,系统难以维持高效的信息交互,从而限制了其向大规模无人机集群协同飞行的扩展能力。

针对上述问题,亟需一种能够在大规模、高密度无人机集群中同时实现高效通信与稳定测距的 UWB 测距协议。
Shan 等人\cite{shan2022ultra,shan2021ultra}提出的集群测距协议在理论上具备实现高效测距与通信的潜力,但该协议最初并非面向相对定位任务设计,在实际应用中暴露出一定的鲁棒性问题。
例如,当测距消息以固定周期发送时,节点间时钟偏差可能导致持续的消息冲突,从而长时间无法完成距离计算;
而当测距消息以完全随机方式发送时,消息交换不匹配将不可避免,由于缺乏有效时间戳,距离计算难以进行。
在定位任务中,连续且频繁的测距失败将直接降低集群中各个无人机的定位精度,且在高密度集群中,该问题会因相邻节点数量增多而进一步加剧。


除面向高密度集群测距场景的研究外,基于超宽带的相关工作还广泛分布于其他室内定位应用场景,主要集中在测距协议设计、非视距(Non-Line-of-Sight,NLOS)环境处理以及测距误差补偿等方面。


在测距协议设计方面,Chiasson 等人 \cite{chiasson2023asynchronous} 提出了一种新的基于到达时间差(Time Difference of Arrival,TDOA)的双曲定位方程。该方法利用能够相互观测数据包到达时间的锚节点,对传统 TDOA 方程进行重构,从而有效降低了时钟漂移误差的影响,并在无需严格时间同步的条件下实现了高精度定位。
Yang 等人 \cite{yang2022vuloc}提出了虚拟双向测距方法,该方法支持大量目标的无同步定位,显著降低了系统对时间同步和校准的依赖。上述研究均在无需严格时间同步的前提下实现了高效测距,但通常依赖固定锚点部署。

在 NLOS 场景处理方面,Zhao 等人 \cite{zhao2022finding} 提出了一种新型算法,在存在障碍物的情况下,通过优化 UWB 锚点的部署位置来降低 NLOS 对定位精度的影响。
Kim 等人 \cite{kim2022uwb} 则利用长短期记忆网络(Long Short-Term Memory,LSTM)对接收到的 UWB 信号信道脉冲响应进行信道状态分类,从而识别 NLOS 状态并提升定位性能。

在测距误差补偿方面,Domuta 等人 \cite{domuta2021two} 提出了两种额外的时间戳捕获机制以及一种新的飞行时间计算公式,用于系统性补偿节点间的时钟偏差。
Cano 等人 \cite{cano2022clock} 在此基础上进一步提出了一种仅依赖板载测量值的补偿方法,能够消除视距 ToF 测量中大部分观测偏差,并可靠地实现厘米级测距精度。
此外,Shalaby 等人 \cite{liu2023relative} 对双向测距的校准问题进行了深入研究,提出了一种全新的测距协议,结合稳健且可扩展的天线延迟校准程序,实现了多个 UWB 标签天线延迟的高效与精确校准。

综上所述,目前尚不存在能够在资源受限的大规模集群中支持大量节点同时进行测距的理想测距协议。现有研究大多面向节点数量有限或依赖固定锚点的应用场景,其协议设计通常基于时隙分配或集中式调度机制,在高密度、动态变化的无人机集群环境中难以同时兼顾测距效率、通信开销与系统鲁棒性。

\subsection{基于距离的相对定位算法}
高效稳定的定位方法是实现无人机之间协同感知与协调控制的关键基础,因此长期以来受到研究人员的广泛关注。
传统无人机的定位通常依赖于外部定位系统。例如,文献\citen{vasarhelyi2018optimized}中的无人机集群系统利用全球导航卫星系统(Global Navigation Satellite System,GNSS)实现对 30 架无人机的定位,从而成功展示了户外集群飞行行为。
然而,微型无人机往往在室内或封闭环境中运行,此类环境中 GNSS 等外部定位信号难以获取甚至完全不可用。
针对这一问题,已有研究提出了多种室内定位系统,通常依赖于预先部署的信标或锚点为空中集群提供定位信息 \cite{jia2022composite,oumar2023indoor,brandstatter2022multi}。
尽管这类方法在受限环境下能够实现较高精度的定位,但其对外部基础设施的依赖限制了在危险区域或未知环境中的应用,实际部署成本和灵活性均受到制约。

作为替代方案,基于无人机间无线通信的相对定位方法因其系统结构轻便、对机载资源需求较低而备受关注。
该类方法通过无线信道交换无人机的状态信息(如速度、偏航角速度和高度),并结合无线测距获取的相对距离信息来实现相对定位 \cite{nguyen2023relative,cao2021relative}。
文献 \citen{nguyen2023relative} 提出了基于优化的协同定位框架,包括二次约束方法及其半正定规划松弛形式。
尽管上述方法在机载无人机上能够获得较高的定位精度,但其单次定位计算时间约为 15 ms,对于计算资源和功耗受限的微型无人机而言仍然构成较大挑战。
Cao 等人 \cite{cao2021relative} 提出了一种利用多个超宽带测距节点估计无人机之间相对位姿的方法,但该方法通常依赖更大尺寸的平台和更强的计算能力,限制了其在微型无人机系统中的应用。
与上述基于优化或多节点测距的方法不同,基于滤波的协同定位方法通过递推状态估计来实现实时定位,其中扩展卡尔曼滤波器(Extended Kalman Filter,EKF)基于一阶线性化,因其结构简单、计算复杂度低,成为协同定位中最常用的状态估计方法之一\cite{li2020autonomous}。

为进一步提升协同定位的精度与系统鲁棒性,研究者相继提出了多种改进型非线性滤波方法,包括无迹卡尔曼滤波(UKF)\cite{allotta2016development}、基于 Huber 损失的鲁棒滤波器\cite{bo2019cooperative}、 基于Student’s t 分布建模的鲁棒高斯近似滤波方法\cite{huang2017new}、粒子滤波算法\cite{akai2023reliable}以及移动视界估计(MHE)方法\cite{wang2023neural}。
然而,上述协同定位算法的性能普遍对系统噪声和测量噪声协方差矩阵的先验设定高度敏感。当噪声统计特性建模存在偏差时,滤波器的估计精度将显著下降,严重情况下甚至可能引发滤波发散问题。
在微型无人机协同定位的复杂动态环境中,由于传感器误差、通信干扰及环境不确定性的共同影响,噪声协方差矩阵往往难以被准确刻画,这在一定程度上制约了现有方法在实际场景中的应用效果。

自适应卡尔曼滤波器(Adaptive Kalman Filter,AKF)被认为是解决无人机集群协同定位中噪声协方差矩阵未知问题的一种有效方法\cite{kruse2025adaptive}。
针对噪声统计特性未知的情形,已有大量相关研究工作,现有方法总体上可分为两类。

第一类自适应卡尔曼滤波方法通常假设过程噪声协方差矩阵已知,
主要致力于对测量噪声协方差矩阵的在线估计,
例如基于变分贝叶斯框架的自适应卡尔曼滤波方法\cite{davari2019variational}以及基于协方差匹配原理的自适应卡尔曼滤波方法\cite{hajiyev2023covariance}。
然而,此类方法对过程噪声协方差矩阵的先验信息依赖较强,当过程噪声的统计特性未知或存在建模偏差时,其滤波性能和适用范围将受到明显制约。

另一类自适应卡尔曼滤波方法针对过程噪声与测量噪声协方差矩阵均未知的情形展开研究。
其中,多模型自适应卡尔曼滤波通过并行运行多组卡尔曼滤波器来应对系统模型的不确定性,但其计算复杂度较高,难以满足实时性要求\cite{youn2020novel};
迭代扩展卡尔曼滤波则利用创新序列的白噪声特性对噪声协方差矩阵进行估计,然而在过程噪声和测量噪声同时未知的情况下,该方法易导致滤波发散,从而影响估计稳定性\cite{lin2022efficient}。

综上所述,现有自适应卡尔曼滤波方法在计算开销、滤波稳定性以及整体性能等方面仍存在一定局限,使其在微型无人机协同定位等对实时性和鲁棒性要求较高的应用场景中面临诸多挑战。

\section{无人机自主导航}

\subsection{依赖地图的自主导航}
\subsection{不依赖地图的自主导航}
\section{研究现状小结}

相比之下,基于无线通信的定位方法不依赖视距条件,具有覆盖范围广、对环境适应性强等优势,更适合用于无人机集群的相对定位任务。
其中,超宽带(Ultra-Wideband,UWB)无线电技术因其纳秒级时间分辨率和抗多路径能力强,已成为当前定位领域的重要研究方向,并被广泛应用于室内定位系统[17–23]以及多机器人系统[25–28]。
超宽带技术能够在无人机之间的数据交换过程中提供精确的发送与接收时间戳,从而支持基于飞行时间(Time of Flight, ToF)的距离测量,实现无人机间距离的连续估计与动态更新。
相比之下,基于无线通信的定位方法不依赖视距条件,具有覆盖范围广、对环境适应性强等优势,更适合用于无人机集群的相对定位任务。
其中,超宽带(Ultra-Wideband,UWB)无线电技术因其纳秒级时间分辨率和抗多路径能力强,已成为当前定位领域的重要研究方向,并被广泛应用于室内定位系统[17–23]以及多机器人系统[25–28]。
超宽带技术能够在无人机之间的数据交换过程中提供精确的发送与接收时间戳,从而支持基于飞行时间(Time of Flight, ToF)的距离测量,实现无人机间距离的连续估计与动态更新。
相比之下,基于无线通信的定位方法不依赖视距条件,具有覆盖范围广、对环境适应性强等优势,更适合用于无人机集群的相对定位任务。
其中,超宽带(Ultra-Wideband,UWB)无线电技术因其纳秒级时间分辨率和抗多路径能力强,已成为当前定位领域的重要研究方向,并被广泛应用于室内定位系统[17–23]以及多机器人系统[25–28]。
超宽带技术能够在无人机之间的数据交换过程中提供精确的发送与接收时间戳,从而支持基于飞行时间(Time of Flight, ToF)的距离测量,实现无人机间距离的连续估计与动态更新。
相比之下,基于无线通信的定位方法不依赖视距条件,具有覆盖范围广、对环境适应性强等优势,更适合用于无人机集群的相对定位任务。
其中,超宽带(Ultra-Wideband,UWB)无线电技术因其纳秒级时间分辨率和抗多路径能力强,已成为当前定位领域的重要研究方向,并被广泛应用于室内定位系统[17–23]以及多机器人系统[25–28]。
超宽带技术能够在无人机之间的数据交换过程中提供精确的发送与接收时间戳,从而支持基于飞行时间(Time of Flight, ToF)的距离测量,实现无人机间距离的连续估计与动态更新。


