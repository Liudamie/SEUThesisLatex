\chapter{绪论}
%\label{chp:installation}

\section{研究背景及意义}
%\label{sec:tex_environment}
\subsection{微型无人机}

近些年来,无人机技术发展迅速,其应用场景不断拓展,逐渐在社会生产与公共管理等领域中发挥重要作用。
相比传统有人作业方式,无人机具备部署迅速、机动性强、任务适应能力高等特点,尤其适合在复杂、危险或人员难以进入的环境中开展作业。
在自然资源调查领域,无人机搭载激光雷达、航测相机等设备,可完成高精度三维建模。
如 2025 年,生态环境部相关团队在锡林郭勒草原保护区,通过无人机多光谱、可见光技术结合地面监测,实现了草原植物多样性的高效监测。
在公共安全与基础设施运维领域,无人机凭借智能感知与识别技术,实现输电线路等设施常态化巡查。
2025 年湖北大冶电网巡检中,无人机大幅提升巡检效率,缩短故障响应时间,实现多个重点台区 “零停电消缺”。
此外,在应急与灾害响应场景中,无人机搭载红外探测、实时通信模块,能够快速的获取现场信息。
2025 年湖南祁东县山林火情处置中,无人机快速定位火情、传输影像,仅用 18 分钟完成信息报送,为快速扑灭火情提供关键支撑。

\begin{figure}[htbp]
    \centering
    \subfloat[地质勘测]{\includegraphics[width=.3\textwidth]{figures/content/chapter1/地质勘测.png}}
    \quad
    \subfloat[电力巡检]{\includegraphics[width=.26\textwidth]{figures/content/chapter1/电力巡检.png}}
    \quad
    \subfloat[电力巡检]{\includegraphics[width=.27\textwidth]{figures/content/chapter1/火灾救援.jpeg}}
    \caption{无人机应用场景}
    %\label{fig:4_4}
    \end{figure}

随着电子制造技术、嵌入式系统以及微机电系统的飞速发展,无人机系统在感知、计算和控制能力不断提升的同时,正朝着小型化、轻量化和智能化方向不断演进。
在此背景下,微型无人机(Micro Unmanned Aerial Vehicles, MUAVs)凭借体积小、重量轻、机动性强等特点,逐渐从科研实验室走向工业应用与日常生活,在工业自动化\cite{awasthi2023micro}、农业\cite{sakya2023cloud}、军事行动\cite{king2024robot}以及灾害响应\cite{ingale2023multi}等领域展现出广阔的应用前景。
\begin{figure}[htbp]
    \centering
    \includegraphics[width=.5\linewidth]{figures/content/chapter1/微型无人机.jpg}
    \caption{微型无人机}
\end{figure}

Bitcraze 公司于 2019 年 2 月发布的 Crazyflie 2.1 是当前微型无人机领域中具有代表性的开源平台之一。
该无人机尺寸仅为 65 mm × 65 mm,重量约 27 g,集成了多种传感器模块,并支持多种无线通信方式,广泛应用于无人机控制算法验证、编队飞行研究以及教育科研实验。
以 Crazyflie 2.1 为代表的微型无人机在实际应用中展现出诸多优势:
首先,其体积小、质量轻,在室内环境中飞行安全性高,对人员和设备的潜在威胁较小;
其次,系统成本低、维护简单,便于实现多机协同与大规模部署;
此外,其开源特性和模块化设计为算法开发、系统扩展和功能定制提供了良好的实验基础。
然而,微型无人机同样存在一定的局限性,例如受限于机载硬件规模,微型无人机的计算与存储资源相对有限,单机可执行任务的复杂度较低;
此外,其电池容量较小,飞行续航时间较短,从而限制了单次任务的执行时间和活动范围,这些不足对系统的稳定性与任务复杂度提出了更高要求。

\subsection{微型无人机集群协作飞行}
受限于单架微型无人机在计算能力、存储能力、续航时间及任务执行效率方面的不足,通过多机协同提升系统整体性能的无人机集群技术逐渐兴起,并成为当前研究的热点方向。
通过多架微型无人机之间的协同配合,集群系统能够在保证单机低成本与高安全性的前提下,实现复杂任务的分布式执行,从而显著提升系统整体的可靠性与任务完成能力。

\begin{figure}[htbp]
    \centering
    \includegraphics[width=.5\linewidth]{figures/content/chapter1/微型无人机集群飞行.jpg}
    \caption{微型无人机集群协作飞行}
\end{figure}

微型无人机集群本质上属于一种典型的分布式系统,其协同飞行与任务执行高度依赖于集群内部的信息交互机制。
在协作过程中,首先需要解决的是微型无人机之间的通信与距离感知问题。
在高密度集群环境中,多机同时通信与测距会导致信道竞争和干扰加剧,使得通信资源更加紧张。
因此,如何在受限的信道环境下实现稳定、低延迟且具有鲁棒性的集群测距,是微型无人机协同飞行面临的关键挑战之一。

在实现集群通信与测距的基础上,精确的相对定位能力是微型无人机集群协作的另一项核心需求。
与依赖外部定位基础设施的传统无人机不同,微型无人机集群通常工作于室内或复杂环境中,难以依赖卫星导航系统获取全局位置信息。
在此情况下,微型无人机之间需要通过无线信号、传感器信息或视觉等方式获取彼此的相对位置关系,为编队保持、队形调整以及协同决策提供必要的空间约束。

此外,由于微型无人机集群在执行任务过程中通常缺乏统一的中心控制节点或领导者,各无人机主要依赖分布式决策进行运动控制,其飞行轨迹具有较强的动态性和不确定性。
这种无领导结构在一定程度上降低了系统的复杂度,但也显著增加了集群在自主导航与安全保障方面的难度。
因此,引入具备全局或半全局信息的领导者无人机,用于对集群整体运动方向和任务目标进行引导,对于提升集群协同飞行的有序性与安全性具有重要意义。
在引入领导者的集群协同飞行模式下,领导者无人机负责进行全局路径规划或目标引导,其余跟随无人机依据与领导者及邻近无人机之间的相对信息实现协同运动。
同时,各无人机仍需实时感知外部环境和集群内部状态,以避免与障碍物或其他无人机发生碰撞,从而在保证集群整体安全性与稳定性的前提下,实现高效、可靠的协同导航与任务执行。

\begin{figure}[htbp]
    \centering
    \subfloat[室内探索搜救]{\includegraphics[width=.38\textwidth]{figures/content/chapter1/室内探索搜救.png}}
    \quad
    \subfloat[战术协同侦测打击]{\includegraphics[width=.41\textwidth]{figures/content/chapter1/战术协同侦测打击.png}}
    \quad
    \subfloat[灾害重建]{\includegraphics[width=.4\textwidth]{figures/content/chapter1/灾害重建.png}}
    \quad
    \subfloat[货物运输]{\includegraphics[width=.4\textwidth]{figures/content/chapter1/货物运输.png}}
    \caption{微型无人机集群应用场景}
    \label{fig:微型无人机集群应用场景}
    \end{figure}

无人机集群内部的鲁棒集群测距、高效定位与自主导航避障技术是集群协作的基础,是实现并保障集群协作高效进行的关键技术,可以应用于众多实际场景中。
如图\ref{fig:微型无人机集群应用场景}所示,有研究人员利用多架搭载高清摄像头的微型无人机组成集群,在复杂的室内环境中搜寻伤者并实时上报伤者位置\cite{pliakos2024preliminary};
微型无人机集群也可用于军事场景中,利用微型无人机体积小、易伪装隐藏的优势,可用于在房屋、小巷、森林等复杂场景中实时侦测敌方目标动态,以辅助制定攻击战术;
微纳型无人机通过搭载信号感应装置,在机场等关键基础设施附近分散协同搜寻恶意干扰信号发射源,并在发现恶意人员后协作跟踪恶意人员的动向并实时上报给安保人员;
有研究人员在微型无人机上搭载激光雷达与通信模块并组成集群在复杂动态的场景中实现了集群的自主导航\cite{fei2024deep}。
因此,研究微型无人机集群协同飞行具有很强的现实意义。

\section{研究思路与挑战}
本研究围绕微型无人机集群的自适应协同定位与自主导航问题展开系统性研究,旨在构建支撑微型无人机集群协同飞行的基础技术体系。
在此基础上,本文将微型无人机集群协同飞行问题进一步划分为鲁棒集群测距、自适应相对定位算法和自主导航三个关键模块。

首先,在通信与测距的基础层面,针对无人机集群中测距冲突频繁、测距失败率较高等问题,本文以超宽带通信技术为基础,研究并设计了一种鲁棒的集群测距协议。
通过对集群测距中超宽带芯片的时钟偏移、报文冲突等关键因素进行理论分析,引入扰动机制并对测距策略进行权衡与优化,从而在复杂动态环境下为集群协同提供稳定可靠的通信与距离观测信息。

其次,在相对定位层面,基于鲁棒集群测距所获得的距离观测与可靠通信,本文进一步研究自适应相对定位算法。针对相对定位过程中误差协方差矩阵未知或随时间变化的问题,通过构建系统模型并引入期望最大化算法,实现对无人机位置状态与噪声误差协方差矩阵的联合估计,从而提升无人机集群在复杂环境下的定位精度与估计稳定性。

最后,在自主导航飞行层面,本文首先面向单架无人机的自主飞行任务,研究基于分布式强化学习的自主导航方法。通过合理设计奖励函数与风险感知机制,构建兼顾实时性与安全性的导航决策模型,并进一步研究模型的轻量化与部署方法,实现自主导航算法在真实无人机平台上的高效运行与实验验证。
在此基础上,本文将所提出的自主导航方法扩展至无人机集群协同飞行场景:由领导者无人机执行自主导航与路径规划任务,其余跟随无人机基于相对定位结果获取与领导者之间的相对位姿信息,实现对领导者的稳定跟随与队形保持,从而完成集群层面的协同飞行任务。

上述研究思路表明,实现微型无人机集群的协同定位与自主导航飞行在技术上是可行的,但在实际应用中仍面临三个主要挑战。

第一个挑战在于如何设计一种具备自适应性与抗冲突能力的通信测距调度机制,以在复杂动态场景中稳定支撑相对定位任务。
已有研究提出的集群测距协议在理论上能够实现高效的集群测距与通信,但其设计初衷并非面向相对定位应用,在实际部署中暴露出明显的鲁棒性不足。一方面,固定周期的测距报文传输方式容易受到节点间时钟偏差的影响,导致持续性的报文冲突,从而使得距离计算在较长时间内无法完成;另一方面,完全随机的报文传输方式又容易造成测距报文交换不匹配,因缺乏有效时间戳而导致测距失败。在高密度群体环境下,相邻节点数量的增加进一步加剧了通信冲突和报文丢失的发生概率,频繁的测距失败将直接削弱各个无人机的相对定位精度。因此,提升集群测距在高密度、动态场景下的稳定性与可靠性,是协同定位面临的首要挑战。

第二个关键挑战在于,在噪声误差协方差矩阵未知或具有时变特性的条件下,如何实现具备稳定性、精度以及自适应能力的相对定位估计。基于集群间测距信息的协同定位通常依赖非线性状态估计方法,其中扩展卡尔曼滤波及其改进算法因计算负担较轻而得到广泛应用。然而,此类方法的性能高度依赖于噪声协方差矩阵的先验设定。
在实际无人机集群定位过程中,测距噪声与运动噪声的统计特性往往难以精确获取,并且会受到环境变化和传感器性能波动的影响而呈现时变特性。当噪声模型存在不准确或失配时,容易导致定位误差累积,甚至引发滤波不稳定或发散现象。因此,在噪声统计信息不确定的情况下实现可靠的相对定位估计,仍是相对定位算法设计中亟待解决的核心问题。

第三个挑战在于如何在计算资源受限的条件下,实现兼顾实时性与安全性的自主导航与避障决策机制。
在完成协同定位的基础上,无人机仍需执行自主导航与避障任务。深度强化学习方法在路径规划问题中表现出较强的优化能力,但其以最大化长期累积回报为目标,往往难以充分关注低概率但高风险的灾难性事件,难以满足复杂动态环境中对飞行安全性的严格要求。同时,该类方法在面对环境突变时的实时响应能力仍存在一定局限。
此外,虽然将推理任务卸载至边缘计算节点可以缓解机载计算压力,但通信延迟与带宽受限可能导致决策结果无法及时反馈,从而增加飞行过程中的安全风险。因此,在微型无人机平台计算资源有限的前提下,如何实现具备实时响应能力和安全保障的自主导航与避障策略,是无人机集群协同飞行中面临的重要挑战。
\section{研究目标与内容}

本硕士论文的总体目标是围绕微型无人机集群协同飞行问题,研究并设计一种无人机集群自适应协同定位算法,以提升无人机集群飞行的稳定性与可靠性。同时,针对无人机自主导航飞行需求,构建基于强化学习的网络模型,探索分布式强化学习在微型无人机自主导航系统中的应用,实现无人机的自主感知与决策控制。

本硕士论文的理论目标是针对纳米级无人机计算资源受限、电池容量有限等特点,重点研究轻量化、低计算复杂度的神经网络模型,在有效避免无人机发生碰撞的前提下,实现无人机向目标位置的自主导航任务。针对无人机集群飞行应用场景,提出一种鲁棒的集群通信与测距机制,并结合自适应扩展卡尔曼滤波方法,设计无人机集群相对定位算法,以保障集群在飞行过程中对各自邻居的相对位姿感知。

本硕士论文的系统目标是结合上述两个方面,设计并实现微型无人机集群飞行原型系统,用于验证所提出的无人机集群自适应协同定位方法及强化学习导航模型的可行性与有效性。

为了实现上述研究目标,本硕士论文针对微型无人机集群协同飞行系统设计与实现问题,拟从下面几个方面展开相关的研究工作:

\begin{figure}[htbp]
    \centering
    \includegraphics[width=.6\linewidth]{figures/content/chapter1/研究内容.png}
    \caption{研究内容}
    \label{fig:研究内容}
\end{figure}

首先,针对无人机集群定位任务部分,需要设计使用与微型无人机集群的通信与定位算法,分两个方面展开:1)基于超宽带的鲁棒集群测距协议;2)基于自适应扩展卡尔曼滤波的相对定位算法。
其次,针对无人机自主导航部分,需要设计构建基于分布式强化学习的神经网络模型,并从以下两个方面展开:1)基于强化学习的神经网络模型构建与训练;2)基于强化学习的神经网络模型部署。

最后,以上述理论研究的基础上,设计开发微型无人机集群飞行系统来实现无人机集群的自主飞行,其中包含无人机集群飞行任务子系统和无人机自主导航飞行任务子系统,其研究内容关系图如图\ref{fig:研究内容}所示。

\section{论文组织结构}
本硕士论文的整体结构安排如下:第一章首先介绍本文的研究背景与研究意义,并围绕微型无人机集群飞行技术分析该领域面临的关键挑战,进而结合相关挑战系统阐述本文的研究内容与研究目标。

第二章阐述本文的研究领域和现有的相关技术,整理并总结了自主避障技术和现有的集群编队控制技术的研究现状,通过分析现有技术的不足,来阐明本硕士论文的创新性。

第三章阐述如何实现视觉辅助单机自主避障飞行的研究。首先本文设计实现基于视觉的多任务网络模型Multi-DroNet,调整各个任务的损失函数,使得网络能够辅助无人机进行转向和碰撞判断,同时输出一个交互信号量,用于控制集群编队形式。并阐述如何实现量化压缩Multi-DroNet模型并成功将其部署到资源受限的微型无人机。同时通过模拟实验和真机实验来验证网络模型的效果,并最终实现单架微型无人机的室内自主避障飞行。

第四章阐述如何设计实现高效的集群交互编队飞行控制算法。首先,本文将面向多智能体设计的集群测距协议应用到无人机集群中,使用基于EKF的相对定位算法根据测距通信协议数据包中的参数进行无人机集群的相对定位。

最后设计基于Leader-Follower编队飞行算法,根据网络模型的交互信号量预测结果实现无人机集群的交互编队避障飞行,使用PID控制策略对Follower飞行控制参数进行误差消除。

第五章阐述原型系统的设计与开发,包括系统的硬件平台搭建、软件模块的架构和开发、系统部署和运行方案以及相应的测试结果分析等。

第六章总结本硕士论文的研究成果以及不足点,并探讨了未来可以改进的方向。