\chapter{绪论}
%\label{chp:installation}

\section{研究背景及意义}
%\label{sec:tex_environment}
\subsection{微型无人机}

近些年来,无人机技术发展迅速,其应用场景不断拓展,逐渐在社会生产与公共管理等领域中发挥重要作用。
相比传统有人作业方式,无人机具备部署迅速、机动性强、任务适应能力高等特点,尤其适合在复杂、危险或人员难以进入的环境中开展作业。
在自然资源调查领域,无人机搭载激光雷达、航测相机等设备,可完成高精度三维建模。
如 2025 年,生态环境部相关团队在锡林郭勒草原保护区,通过无人机多光谱、可见光技术结合地面监测,实现了草原植物多样性的高效监测。
在公共安全与基础设施运维领域,无人机凭借智能感知与识别技术,实现输电线路等设施常态化巡查。
2025 年湖北大冶电网巡检中,无人机大幅提升巡检效率,缩短故障响应时间,实现多个重点台区 “零停电消缺”。
此外,在应急与灾害响应场景中,无人机搭载红外探测、实时通信模块,能快速获取现场信息。
2025 年湖南祁东县山林火情处置中,无人机快速定位火情、传输影像,仅用 18 分钟完成信息报送,为快速扑灭火情提供关键支撑。

\begin{figure}[htbp]
    \centering
    \subfloat[地质勘测]{\includegraphics[width=.3\textwidth]{figures/content/chapter1/地质勘测.png}}
    \quad
    \subfloat[电力巡检]{\includegraphics[width=.26\textwidth]{figures/content/chapter1/电力巡检.png}}
    \quad
    \subfloat[电力巡检]{\includegraphics[width=.27\textwidth]{figures/content/chapter1/火灾救援.jpeg}}
    \caption{无人机应用场景}
    %\label{fig:4_4}
    \end{figure}

随着电子制造技术、嵌入式系统以及微机电系统的飞速发展,无人机系统在感知、计算和控制能力不断提升的同时,正朝着小型化、轻量化和智能化方向不断演进。
在此背景下,微型无人机(Micro Unmanned Aerial Vehicles, MUAVs)凭借体积小、重量轻、机动性强等特点,逐渐从科研实验室走向工业应用与日常生活,在工业自动化\cite{awasthi2023micro}、农业\cite{sakya2023cloud}、军事行动\cite{king2024robot}以及灾害响应\cite{ingale2023multi}等领域展现出广阔的应用前景。
\begin{figure}[htbp]
    \centering
    \includegraphics[width=.5\linewidth]{figures/content/chapter1/微型无人机.jpg}
    \caption{微型无人机}
\end{figure}

Bitcraze 公司于 2019 年 2 月发布的 Crazyflie 2.1 是当前微型无人机领域中具有代表性的开源平台之一。
该无人机尺寸仅为 65 mm × 65 mm,重量约 27 g,集成了多种传感器模块,并支持多种无线通信方式,广泛应用于无人机控制算法验证、编队飞行研究以及教育科研实验。
以 Crazyflie 2.1 为代表的微型无人机在实际应用中展现出诸多优势:
首先,其体积小、质量轻,在室内环境中飞行安全性高,对人员和设备的潜在威胁较小;
其次,系统成本低、维护简单,便于实现多机协同与大规模部署;
此外,其开源特性和模块化设计为算法开发、系统扩展和功能定制提供了良好的实验基础。
然而,微型无人机同样存在一定的局限性,例如受限于机载硬件规模,微型无人机的计算与存储资源相对有限,单机可执行任务的复杂度较低;
此外,其电池容量较小,飞行续航时间较短,从而限制了单次任务的执行时间和活动范围,这些不足对系统的稳定性与任务复杂度提出了更高要求。

\subsection{Apple MacOS\texttrademark}

\LaTeX 在Apple MacOS操作系统上的发行版称为MacTeX。在 MacOS 上安装 MacTeX 之前,请确保你已经正确安装了\href{https://brew.sh/}{homebrew}。当然,你也可以直接从\href{http://www.tug.org/mactex/index.html}{官网}下载 MacTeX套件,但本文建议你使用 homebrew 安装纯净的 MacTeX 发行版。MacTeX 分为基本版和完全版,区别主要在于完全版中默认包含了更多的宏包。安装基本版 MacTeX 已经可以应付你绝大多数 \LaTeX 需求,在终端中输入:

\begin{tcolorbox}
\begin{lstlisting}
brew cask install basictex
\end{lstlisting}
\end{tcolorbox}

\noindent 你就可以获得了基本版的 MacTeX。如果你一定要安装完全版,请在终端中输入:

\begin{tcolorbox}
\begin{lstlisting}
brew cask install mactex
\end{lstlisting}
\end{tcolorbox}

\subsection{Ubuntu Linux}

在 Ubuntu 中配置 \LaTeX 开发环境最为简单。事实上如果你是一个 GNU/Linux 使用者,你应该已经具有了相当的工程能力能够自行配置 \LaTeX 编译环境。但为了本文结构上的完整,我们决定还是多此一笔。在终端中输入:

\begin{tcolorbox}
\begin{lstlisting}
sudo apt install texlive-full
\end{lstlisting}
\end{tcolorbox}

\noindent 你就可以在 Ubuntu 设备上部署 Tex Live 发行版。其他 Linux 发行版上的安装方法与 Ubuntu Linux 类似,只是各自使用的包管理器可能有所不同,请参阅各发行版的包管理中心网站,本文不再赘述。

\section{模板的下载与安装}
\label{sec:template_download}

其实在你看到本手册的同时,我们相信你已经成功地将本模版下载到了你的设备上。因此本来并没有必要在此赘述介绍工程的下载方法。但为了防止你下载的并非最新版本的模板工程,或者本模板被其他网站转载而你恰好从别的网站上下载了本模板,我们觉得还是有必要介绍一下我们指定的下载地址。本模板工程的所有代码都已经在GitHub上开源,你可以从\href{https://github.com/herculas/SEU-master-thesis}{这个地址}找到本模板的最新版本。

将本模板工程文件解压缩到你喜欢的目录下,你就得到了完整的模板工程。为了避免不必要的编译问题,我们建议你将工程保存在全英文的目录下。本模板已在 Windows 10,MacOS 10.15 Catalina,Ubuntu 18.04 Bionic Beaver以及Manjaro 19.0.2 上编译通过,但需要注意的是一些 Linux 发行版中没有安装本模板编译所需的字体文件,如宋体、黑体、楷体和 Times New Roman 等。因此如果你在 Linux 下遭遇了编译问题,请首先检查你的字体是否都已经安装完好。

\section{论文的编译}
\label{sec:compilation}

如果你使用的是如 Tex Studio,Texpad 或 WinEdt 等 \LaTeX 集成环境,你可以从这些软件中直接启动编译。但是作为一个较为庞大的、涉及多文件的 \LaTeX 工程,你可能需要多次编译才能获得完整的论文。一个完整的编译过程包含下面几个步骤:

\begin{tcolorbox}
\begin{lstlisting}
xelatex main
bibtex main
makeindex main.nlo -s nomencl.ist -o main.nls
xelatex main
xelatex main
\end{lstlisting}
\end{tcolorbox}

\noindent 想要编译一篇学位论文,首先需要对文章结构和原始文本进行一次预编译;随后索引出论文中出现的所有参考文献,并建立参考文献条目与论文引用位置的连接;接下来,根据预编译所产生的文章结构,需要生成文章的图表和术语索引文件;最后通过两次编译将参考文献和图表索引编入正文中,得到完整的PDF版本论文。可以看到这个过程极其复杂,因此我们为你准备了两个脚本文件,来将你从复杂的编译流程中解脱出来。对于 Windows 用户,你可以双击工程根目录下的 make.bat 文件启动编译流程。而 MacOS 和 Linux 用户则可以在命令行中执行根目录下的 make.sh 脚本来启动编译流程。

对于使用类 Unix 操作系统的用户,我们也在 3.4.3 版本之后加入了对 GNU Make 的支持,你现在可以使用 make 命令进行论文的自动化增量编译。GNU Make 工具和 make 命令的相关知识,可以参考\href{https://www.gnu.org/software/make/manual/make.html}{这里}。
